%\VignetteIndexEntry{gprege Quick Guide}
%\VignetteKeywords{TimeCourse, GeneExpression, Transcription, DifferentialExpression}
%\VignettePackage{gprege}
\documentclass[a4paper]{article}
\usepackage{url}
\usepackage[authoryear,round]{natbib}

\title{gprege Quick Guide}
\author{Alfredo A. Kalaitzis, Neil D. Lawrence and Antti Honkela}

\newcommand{\Rfunction}[1]{{\texttt{#1}}}
\newcommand{\Robject}[1]{{\texttt{#1}}}
\newcommand{\Rpackage}[1]{{\textit{#1}}}
\newcommand{\gprege}{\Rpackage{gprege}}

\usepackage{Sweave}
\begin{document}
\maketitle


\section{Abstract}
 

The \gprege{} package implements our methodology of Gaussian process
regression models for the analysis of microarray time series.
The package can be used to filter quiet genes and quantify differential expression in time-series expression ratios or for ranking candidate targets of a trascription factor.

The purpose of this quick guide is to present a few examples of using gprege.

\section{Citing \gprege{}}

Citing \gprege{} in publications will usually involve citing the methodology paper \citep{Kalaitzis:simple11} that the software is based on as well as citing the software package itself.


\section{Introductory example analysis - TP63 microarray data}
\label{section:Introductory example}

In this section we introduce the main functions of the \Rpackage{gprege}
package by repeating some of the analysis from the BMC Bioinformatics
paper~\citep{Kalaitzis:simple11}

\subsection{Installing the \gprege{} package}

The recommended way to install \gprege{} is to use the
\Rfunction{biocLite} function available from the bioconductor
website. Installing in this way should ensure that all appropriate
dependencies are met.

\begin{Schunk}
\begin{Sinput}
> source("http://www.bioconductor.org/biocLite.R")
> biocLite("gprege")
\end{Sinput}
\end{Schunk}

Otherwise, to manually install the gprege software, unpack the software and run
\begin{verbatim}
  R CMD INSTALL gprege
\end{verbatim}

To load the package start R and run
\begin{Schunk}
\begin{Sinput}
> library(gprege)